\documentclass{article}
\usepackage[ngerman]{babel}
\usepackage[a4paper,margin=2cm]{geometry}
\usepackage{parskip}
\usepackage{graphicx}
\usepackage{amsmath,amscd}

\title{Elliptische Kurven: Was ist das? \\
  Und wie macht man damit Kryptographie?}
\author{Nils Rennebarth}
\date{November 2024}

\newcommand{\C}{\mathbb{C}}
\newcommand{\F}{\mathbb{F}}
\newcommand{\N}{\mathbb{N}}
\newcommand{\Proj}{\mathbb{P}}
\newcommand{\Q}{\mathbb{Q}}
\newcommand{\R}{\mathbb{R}}
\newcommand{\Z}{\mathbb{Z}}
\newcounter{thm}
\newtheorem{folgerung}[thm]{Folgerung}
\newtheorem{beispiel}[thm]{Beispiel}
\begin{document}
\maketitle
\section{Einführung}
Verlauf des Vortrages:

\begin{itemize}
\item
  Was sind sie, wie sehen sie anschaulich aus?
\item
  Addition geometrisch gesehen -> Formeln
\item
  Übergang zur projektiven Ebene
\item
  Um mit Computern zu rechnen ohne Rundungsfehler -> Restklassen
\item
  Wie zur Kryptographie nutzen (Diffie-Hellmann für elliptische Kurven)
\item
  Fun fact: Ursprünglich erforscht um RSA zu attackieren
\item
  Fun fact: Elliptische Kurven können als zweidimensionale
  Verallgemeinerung zum eindimensionalen Kreis angesehen werden
\end{itemize}

In Abbildung \ref{fig:ecurves} erst man ein paar Beispiele für elliptische Kurven.
Genauer gesagt sind dies elliptische Kurven in der affinen rellen Ebene.


\begin{figure}[h]
  \includegraphics[height=5cm]{ec1-m2-p2.png}
  \includegraphics[height=5cm]{ec2-m2-p1.png}
  \includegraphics[height=5cm]{ec2-m2-p1.png}
  
  \caption{Elliptische Kurven}
  \label{fig:ecurves}
\end{figure}

\begin{figure}
  \centering
    \includegraphics[height=5cm]{nec-m3-p2.png}
  \caption{Keine elliptische Kurve}
  \label{fig:noecurve}
\end{figure}

Was haben die mit (mathematischen) Ellipsen zu tun?

\begin{itemize}
\item
  Die Berechnung der Bogenlänge einer Ellipse, genauer: Das Problem aus dem
  Winkel die Länge des Bogens bis dorthin zu berechnen, führt auf Integrale
  die sich nicht in geschlossener Form lösen lassen. Besonders frustrierend,
  da das ganze beim Kreis so gut funktioniert, und die Verallgemeinerung von
  Kreis auf Ellipse bei der Flächenberechnung kein Problem darstellt.
\item
  Die Umkehrfunktion davon kann man in die komplexen Zahlen hinein
  fortsetzen. Nennt man elliptische Funktionen.
\item
  Mit so einer Funktion und ihrer Ableitung kann man eine 1-1 Abbildung
  von den Punkten einer elliptischen Kurve auf ein Parallelogramm der
  komplexen Ebene konstruieren.
\item
  $\ldots$ also eher entfernte Verwandschaft.
\end{itemize}

Kurven von Grad 2 sind alle bekannt und vollständig beschrieben
(Kegelschnitte). Das nächstkompliziertere wären Kurven vom Grad 3.
Die allgemeine Form ist ein davon ist ein Polynom in x,y vom Grad 3,
also:

\begin{equation*}
  a y^3 + b y^2 x + c y x^2 + d x^3 + e y^2 + f x y + g x^2
  + h x + i y + j = 0
\end{equation*}

Das sind 10 verschiedene Parameter, ziemlich viele, sehr
unübersichtlich. Wir wollen eigentlich zwei Kurven die durch Drehung
ineinander übergehen nicht wirklich als unterschiedlich betrachten. Auch etwas
schräg ansehen oder Stauchen und Strecken soll als `dieselbe` Kurve betrachtet
werden. Arithmetisch gesehen heißt das: Es gibt eine Abbildung

\begin{equation*}
  x, y \rightarrow f(x, y), g(x, y)
\end{equation*}

die jeden Punkt (x, y) auf der einen Kurve in einen Punkt (x', y') auf der
anderen Kurve umrechnet, und umgekehrt. Damit lässt sich (kompliziert, braucht
viel Mathematik) zeigen, dass es genügt, Kurven der Form:

$$ y^2 = x^3 + ax + b \text{\quad mit \quad } \Delta_E = -4a^3 - 27b^2 \ne 0 $$

zu betrachten. Diese Form nennt man die \textbf{Weierstraßsche Normalform}.
Die Bedingung ist dazu da, Sachen wie in Abbildung \ref{fig:noecurve}
auszuschließen.

Die aufmerksame Zuhörerin wird sich fragen, warum die Koeffizenten der
Beispiele eigentlich alles ganze Zahlen sind, wenn hier doch ständig von
reellen Zahlen die Rede ist. Mit rationalen Zahlen (vulgo: Brüche ganzer
Zahlen) kann man alle reellen Zahlen genügend genau annähern.
Beliebige relle Zahlen hingegen bestehen aus unendlich vielen unregelmäßigen
Nachkommastellen, mit denen kann man nur ungefähr rechnen, es gibt immer
Rundungsfehler. Wir wollen - kommt gleich - aber eine einzige Operation immer
wieder wiederholen und zwar so richtig oft, da müssen wir ohne sich
akkumulierende Rundungsfehler auskommen, also rechnen wir mit Brüchen ganzer
Zahlen. Die haben das nächste Problem, die werden potentiell - und auch in der
Praxis - immer länger, und unendlich viel Speicherplatz und Rechenkapazität
haben wir leider nicht zur Verfügung. Lösung kommt später.

Vorerst mal bleiben wir bei rationalen Zahlen.

\begin{itemize}
\item
  Fun fact: Die Griechen oder genauer Diophant von Alexandria, betrachtete
  überhaupt nur Zahlen, die sich als ganzzahlige Verhältnisse darstellen
  lassen, also in heutiger Sprache: rationale Zahlen. Er kam dabei auf die
  ersten Probleme die sich - in der heutigen mathematischen Sprache formuliert -
  damit befassten, Punkte auf elliptischen Kurven zu finden, bei denen beide
  Koordinaten rational sind.
\item
  Fun fact negative Zahlen: Diophant kannte bereits negative Zahlen,
  und wusste wie man mit ihnen rechnen musste, hat sie aber nur als
  Hilfsgrößen die bei Zwischenrechnungen auftreten betrachtet und
  nicht als Lösungen von Gleichungen akzeptiert. Bis
  ins 18. Jahrhundert waren Mathematikern negative Zahlen
  suspekt:
  \begin{quote}
    [negative numbers] darken the
  very whole doctrines of the equations and make dark of things which
  are in their nature excessively obvious and simple.
  \hfill Francis Maseres, 1758
  \end{quote}
  Erst im 19. Jahrhundert durch Mathematiker wie Hamilton und Gauß
  etablierten sich negative Zahlen wie wir sie heute kennen.
\end{itemize}

\section{Addition von Punkten auf elliptischen Kurven}

Nehmen wir einfach mal an, wir haben eine elliptische Kurve, gegeben in der
Weierstraßschen Normalform und mit rationalen
Koeffizenten. Nehmen wir weiter an, wir kennen schon zwei Punkte $(x_1, y_1)$
und $(x_2, y_2)$ auf
der Kurve. Dann können wir eine Gerade durch diese legen und bekommen
einen dritten Schnittpunkt.

\begin{figure}[h]
  \centering
  \includegraphics[height=8cm]{ec5-m1-p1-add.png}
  \caption{Addition zweier Punkte P und Q}
  
\end{figure}

Bemerkung 1: Wir bekommen auf keinen Fall einen vierten Schnittpunkt, das
liegt daran, dass wir uns auf Kurve vom Grad 3 beschränkt haben. Warum
können Polynome vom Grad n höchstens n Nullstellen besitzen? Übungsaufgabe.

Tip: Polynomdivision, Nullstelle ist linearer Teiler.

Antwort: Polynomdivision, spalte bei Nullstelle $x_0$ $(x - x_0)$ ab, das
geht genau wenn $x_0$ Nullstelle ist. Bei $n$ Nullstellen komplett zerlegt,
weitere Nullstelle gibt Widerspruch.

Der Beweise dafür nutzt die Tatsache dass alle $x$-Werte der Schnittpunkte einer
Geraden mit der Kurve eine Gleichung dritten Gerades erfüllen, und man für
jede Lösung $x_0$ einen Linearfaktor aus der Gleichung herausziehen kann.

Bemerkung 2: Der dritte Schnittpunkt hat ebenfalls rationale Koordinaten.

Der Beweis benutzt wieder die in Bemerkung 1 erwähnte Zerlegung der Gleichung
für die x-Koordinaten in Linearfaktoren:

   $$ x^3 + a x^2 + b x + d = (x - x_1) (x - x_2) (x - x_3) $$

Ausmultiplizieren der rechten Seite gibt für den Koeffizenten $a$ bei $x^2$:

  $$ a = - (x_1 + x_2 + x_3) $$

Sowohl $x_1$, $x_2$ und $a$ sind nach unseren Annahmen rational, dann muss es
auch $x_3$ sein. $y_3$ liegt auf der Geraden, ist also von der Form
$y = mx + g$ mit $m$ und $g$ beides rational, ist also ebenfalls rational.
  
Damit ist etwas interessantes passiert: Wir haben aus zwei rationalen Punkten
einen dritten konstruiert. Wir können sogar eine Formel für den dritten Punkt
angeben:

Seien $P=(x_p, y_p)$ und $Q=(x_q, y_q)$, $P + Q = R = (x_r, y_r)$, wobei
$s := (y_p - y_q) / (x_p - x_q)$

Dann ist:

\begin{eqnarray*}
  x_r &=& s^2 - x_p - x_q \\
  y_r &=& - y_p + s (x_p - x_r)
\end{eqnarray*}

Und es kommt noch besser: Nicht nur dass wir einen dritten Punkt gefunden
haben, wir haben falls $y_3 \ne 0$ sogar noch einen vierten Punkt, nämlich den
Punkt $(x_4, y_4) = (x_3, -y_3)$, schließlich ist die Kurve in Weierstraß-Normalform
spiegelsymmetrisch zur x-Achse. Und mit diesem vierten Punkt können wir
dieselbe Konstruktion fortsetzen: Gerade durch $(x_1, y_1)$ und $(x_4, y_4)$
legen, wir erhalten einen dritten Schnittpunkt, $(x_5, y_5)$, spiegeln ihn an
der x-Achse und erhalten $(x_6, y_6)$, und so weiter.

\end{document}
