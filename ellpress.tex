\documentclass{beamer}
% Use lualatex to process this file
\usepackage{svg}
\usepackage{german}
\usepackage{amsmath}
\usetheme{Hannover}
\title{Mathematik elliptischer Kurven}
\author{Nils Rennebarth}
\date{November 2024}
\newcommand{\Proj}{\mathbb{P}}
\begin{document}

\begin{frame}
  \titlepage
\end{frame}

\section*{Inhalt}
\begin{frame}
  \tableofcontents
\end{frame}

\section{Geometrie}
\begin{frame}

  \frametitle{Was sind elliptische Kurven}

  \begin{definition}
    Eine elliptische Kurve $E$ über einem Körper $K$ ist eine glatte Kubik im
    projektiven Raum $\Proj^2(K)$ mit mindestens einem Punkt aus $K$.
  \end{definition}

  \pause

  Beispiele:

  \includesvg{ECClines-3}

\end{frame}

\begin{frame}
  \frametitle{Was haben die mit Ellipsen zu tun?}
  \begin{itemize}
  \item Die Berechnung der Bogenlänge einer Ellipse führt auf Integrale die
    sich nicht in geschlossener Form lösen lassen.
  \item Die Umkehrfunktion davon kann man in die komplexen Zahlen hinein
    fortsetzen. Nennt man elliptische Funktionen.
  \item Mit so einer Funktion und ihrer Ableitung kann man eine 1-1 Abbildung
    von den Punkten einer elliptischen Kurve auf ein Parallelogramm der
    komplexen Ebene konstruieren.
  \item \dots{} also eher entferntere Verwandtschaft.
  \end{itemize}
\end{frame}

\begin{frame}
  \frametitle{Wie bekommt man die Gleichung?}
  \begin{itemize}
  \item Wir beginnen mit einem allgemeinen Polynom in zwei Variablen vom
    Grad 3:
    $$ a_0 y^2 + a_1 xy + a_2 y = a_5 x^3 + a_3 x^2 + a_4 x + a_6 $$
  \item Mit $y \to a_0 a_5^2 y $ und $x \to a_0 a_5 x $ wird das zu
    $$ a_0^3 a_5^4 y^2 + \dots = a_0^3 a_5^4 x^3 + \dots $$
  \item Beide Seiten durch $a_0^3 a_5^4$ teilen gibt:
    $$ y^2 + a_1 xy + a_2 y = x^3 + a_3 x^2 + a_4 x + a_6 $$
  \item Mit $y \to y - (a_1 x + a_3)/2$ erhalten wir:
    $$ y^2 = x^3 + a_2 x^2 + a_3 x + a_6 $$
  \end{itemize}
\end{frame}

\begin{frame}
  \frametitle{Weierstraß Normalform}
  \begin{itemize}
  \item Nun noch $x \to x - a_2 / 3$ dann bekommen wir
    \begin{eqnarray*}
      x^3 & - & 3x^2 a_2/3 + 3x a_2^2 / 9 + a_2^3 / 27 \\
          & + & a_2 (x^2 - 2 x a_2 / 3 + a_2^2 / 9) + \dots \\
      = x^3 & + & x^2 (-a_2 + a_2) + x (\dots) + \dots
    \end{eqnarray*}
  \item Und damit auf die bekannte Weierstraß Normalform:
    $$ y^2 = x^3 - a x + b $$
  \end{itemize}
\end{frame}

\begin{frame}
  \frametitle{Addition von Punkten 1}

  \begin{columns}[T]
    \begin{column}{5cm}
      Kurvengleichung:
      
      $$ y^2 = x^3 + a x + b $$

      $P = (x_1, y_1)$ und $Q = (x_2, y_2) $
      Geradengleichung:
      $$ \frac{y - y_1}{x - x_1} = \frac{y_2 - y_1}{x_2 - x_1} =: d $$

      Einsetzen in Kurve gibt Gleichung 3. Grades. Aber wir kennen
      ja schon zwei Lösungen: $x_1$ und $x_2$.
      
    \end{column}
    \begin{column}{5cm}
      \includesvg{add-normal}
    \end{column}
  \end{columns}

%      Kann man nach y auflösen und in die Kurvengleichung einsetzen. Das gibt
%      eine Gleichung dritten Grades in $x$. Aber wir kennen ja schon zwei
%      Lösungen: $x_1$ und $x_2$. Nun gilt allgemein:
\end{frame}
\begin{frame}
  \frametitle{Addition von Punkten 2}
  Allgemein gilt für Gleichung 3. Grades mit Lösungen
  $\lambda_1, \lambda_2, \lambda_3$
  \begin{multline*}
    (x - \lambda_1) (x - \lambda_2 ) (x - \lambda_3) = \\
    \shoveleft{\quad x^3 - (\lambda_1 + \lambda_2 + \lambda_3) x^2
    + (\lambda_1 \lambda_2 + \lambda_1 \lambda_3 +
      \lambda_2 \lambda_3) x - \lambda_1 \lambda_2 \lambda_3}
  \end{multline*}

  Setzen wir in die Kurvengleichung
  $$ 0 = x^3 + ax + b - y^2 $$
  $$y = x d + y_1 - x_1 d$$
  ein und betrachten nur den Koeffizienten von $x^2$, dann ist dieser
  gleich $-d^2$, also ist
  \begin{align*}
    - d^2  & = - (x_1 + x_2 + x_3) \\
    \intertext{und damit:}
    x_3    & = d^2 - x_1 - x_2 \\
    y_3    & = x_3 d + y_1 - x_1 d
  \end{align*}
\end{frame}

\section{Projektive Ebene}


\section{Endliche Körper}


\section{Edwards Kurven}

\subsection{Ed25519}

\begin{frame}
  
\end{frame}
\end{document}
