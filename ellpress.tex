\documentclass{beamer}
% Use lualatex to process this file
\usepackage{svg}
\usepackage{german}
\usepackage{amsmath}
\usetheme{Hannover}
\title{Elliptische Kurven: Was ist das?}
\subtitle{Und wie macht man damit Kryptographie?}
\author{Nils Rennebarth}
\date{November 2024}
\newcommand{\Proj}{\mathbb{P}}
\begin{document}

% ----------------------
\begin{frame}
  \titlepage
\end{frame}
\section{Einführung}
% ----------------------
\begin{frame}

  \frametitle{Was sind elliptische Kurven}
  \framesubtitle<1-4>{Beispiel}
  \framesubtitle<5>{Dies ist \emph{keine} elliptische Kurve}
  \begin{figure}
  \includegraphics<1>[height=0.7\textwidth]{ec1-m2-p2.png}
  \includegraphics<2>[height=0.7\textwidth]{ec2-m2-p1.png}
  \includegraphics<3>[height=0.7\textwidth]{ec3-p1-p0.png}
  \includegraphics<4>[height=0.7\textwidth]{ec4-p0-p4.png}
  \includegraphics<5>[height=0.7\textwidth]{nec-m3-p2.png}
  \end{figure}

\end{frame}


% ----------------------
\begin{frame}
  \frametitle{Was haben die mit Ellipsen zu tun?}
  \begin{itemize}
  \item Die Berechnung der Bogenlänge einer Ellipse führt auf Integrale die
    sich nicht in geschlossener Form lösen lassen.
  \item Die Umkehrfunktion davon kann man in die komplexen Zahlen hinein
    fortsetzen. Nennt man elliptische Funktionen.
  \item Mit so einer Funktion und ihrer Ableitung kann man eine 1-1 Abbildung
    von den Punkten einer elliptischen Kurve auf ein Parallelogramm der
    komplexen Ebene konstruieren.
  \item \dots{} also eher entferntere Verwandtschaft.
  \end{itemize}
\end{frame}

\begin{frame}
  \frametitle{Normalform}
  Allgemeine Kubik:

  \begin{equation*}
    a y^3 + b y^2 x + c y x^2 + d x^3 + e y^2 + f x y + g x^2 +
    h x + i y + j = 0
  \end{equation*}

  Es reicht jedoch, Kurven der folgenden Form zu betrachten:

  \begin{definition}[Weißerstraß Normalform]
    Die Normalform einer elliptischen Kurve E über den reellen Zahlen
    $\mathbb{R}$ ist:

    \begin{equation}
      y^2 = x^3 + ax + b
      \label{eq:weier}
    \end{equation}

    mit a und b aus $\mathbb{R}$, wobei:
    \begin{equation}
      \Delta_E = -4a^3 - 27b^2 \ne 0
    \end{equation}
  \end{definition}

\end{frame}
\section{Addition von Punkten}
\begin{frame}
  \frametitle{Addition von Punkten, geometrisch}
  \begin{figure}
    \includegraphics[height=0.7\textwidth]{ec5-m1-p1-add.png}
  \end{figure}
\end{frame}

\begin{frame}
  \frametitle{Addition von Punkten, mehrfach}
  \begin{figure}
    \includegraphics[height=0.7\textwidth]{ec6-m1-p1-points.png}
  \end{figure}
\end{frame}

\begin{frame}
  \frametitle{Addition von Punkten, arithmetisch}
  \begin{theorem}[Additionsformel]
    Seien $P = (x_p, y_p)$ und $Q=(x_q, y_q)$ zwei Punkte auf der elliptischen
    Kurve $E$. Sei weiterhin:
    \begin{equation*}
      s = \frac{y_p - y_q}{x_p - x_q} \text{ falls } x_p \ne x_q \text{, }
      s = \frac{3x_p^2 + a}{2y_p} \text{ falls } x_p = x_q, y_p \ne 0
    \end{equation*}
    mit $a$ aus (\ref{eq:weier}).
    Dann gilt für den Punkt $R = P \oplus Q = (x_r, y_r)$:
    \begin{equation}
      \begin{split}
        x_r & = s^2 - x_p - x_q \\
        y_r & = -y_p + s(x_p - x_r)
      \end{split}
    \end{equation}
  \end{theorem}
\end{frame}

\begin{frame}
  \frametitle{Additionsgesetze}
  Seien $P$ und $Q$ und $R$ Punkte auf einer elliptischen Kurve, sei $\oplus$
  die Addition von Punkten, dann gilt:
  \begin{equation}
    P \oplus Q = Q \oplus P
  \end{equation}
  \begin{equation}
    (P \oplus Q) \oplus R = P \oplus (Q \oplus R)
  \end{equation}
\end{frame}

\section{Projektive Ebene}
\section{Endliche Körper}
\section{Diffie-Hellmann Schlüsselaustausch}
\begin{frame}
  \tableofcontents
\end{frame}

\end{document}
